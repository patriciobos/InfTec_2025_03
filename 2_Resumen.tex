\begin{center}
	\large{\textbf{\textcolor{black}{Rutinas de configuración y adquisición para anemómetro Gill WindSonic.}}}\\
	\vspace{1pc}
	\textbf{Marcos Remotti y Patricio Bos}
\end{center}

\bigskip
	\begin{center}
		\textcolor{black}{RESUMEN}	
	\end{center}
	\begin*
		\indent
		\textit{El presente informe técnico documenta el diseño e implementación de un módulo de software para configurar y controlar un anemómetro Gill WindSonic. Este instrumento forma parte del equipamiento de la Estación Autónoma Marítima para el Monitoreo de Ruido Ambiente (EAMMRA), desarrollado por la División Acústica Submarina de la Dirección de Investigación de la Armada (DIIV). El software desarrollado permite la adquisición y registro de datos de viento de manera automática o manual, mediante un archivo de argumentos configurables por el usuario.
Este trabajo es parte del Proyecto ``Monitoreo acústico de niveles de ruido submarino y telemetría satelital para detección de eventos acústicos intensos en aguas de la Plataforma Continental Argentina'', del programa PIDDEF del Ministerio de Defensa, vinculado al Proyecto ``Localización e identificación de fuentes de ruido'' de la Armada Argentina, llevado a cabo en la División Acústica Submarina de la Dirección de Investigación de la Armada (DIIV) y la Unidad Ejecutora de Investigación y Desarrollo Estratégicos para la Defensa (UNIDEF), dependiente de CONICET/MinDef.}
	\end*

\bigskip
	\begin{center}
		\textcolor{black}{ABSTRACT}
	\end{center}
	\begin*
		\indent
		\textit{This technical report documents the design and implementation of a software module for configuring and controlling a Gill WindSonic anemometer. This instrument is part of the equipment of the Autonomous Maritime Station for Ambient Noise Monitoring (EAMMRA), developed by the Underwater Sound Division of the Argentinian Navy Research Office (DIIV). The developed software allows for the acquisition and recording of wind data automatically or manually, through a user-configurable argument file.
This work is part of the Project ``Acoustic monitoring of underwater noise levels and satellite telemetry for detecting intense acoustic events in waters of the Argentine Continental Shelf'', from the PIDDEF program of the Ministry of Defense, linked to the Project ``Localization and identification of noise sources'' of the Argentine Navy, carried out in the Underwater Sound Division of the Argentinian Navy Research Office (DIIV) and the Strategic Research and Development Unit for Defense (UNIDEF), dependent on CONICET/MinDef.} 
	\end*

\clearpage
