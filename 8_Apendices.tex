\appendix
\appendixpage
\addappheadtotoc
\section{Desarrollo de códigos en MATLAB para visualización de esferoides prolados y oblados.}
\label{pendiceA}
Un esferoide prolado $E\subset R^3$  puede quedar descripto por dos parámetros. Éstos pueden ser: 
Los Semiejes mayor y menor, $a$ y $b$, siendo a la máxima distancia desde el origen de coordenadas sobre
el eje $x$, y $b$ la máxima distancia desde el origen sobre el eje $z$. 
\[
\frac{x^2}{a^2} + \frac{y^2}{a^2} + \frac{z^2}{b^2} =1
\]



% Viejo método de apéndice es poner un comando \appendix antes del include correspondiente a este archivo para luego en el archivo '8_Apendices.tex' tipear cada sección del apéndice así:
%
%		\addcontentsline{toc}{section}{APÉNDICE I}
%		\section*{APÉNDICE I}
%		\subsection*{{Desarrollo de códigos en MATLAB para visualización de esferoides prolados y oblados.}
%		\label{pendiceA}
